\begin{problem}{Lines}{lines.in}{lines.out}{1 секунда}{64 мегабайта}

В таблице $N \times N$ столбцовнекоторые клетки заняты шариками, другие свободны. Выбран шарик, 
который нужно переместить, и место, куда его нужно переместить. Выбранный шарик за один шаг 
перемещается в соседнюю по горизонтали или вертикали свободную клетку. Требуется выяснить, 
возможно ли переместить шарик из начальной клетки в заданную, и если возможно, то найти путь 
из наименьшего количества шагов.

\InputFile

В первой строке входного файла находится число $N$ ($1 < N \le 250 $), в следующих $N$ строках~-- по $N$ символов. 
Символом точки обозначена свободная клетка, латинской заглавной \t{O}~-- шарик, \t{@}~-- исходное положение шарика, 
который должен двигаться, латинской заглавной \t{X}~-- конечное положение шарика.

\OutputFile

В первой строке выходного файла выводится \t{Y}, если движение возможно, или \t{N}, если нет. 
Если движение возможно, далее следует N строк по N символов - как и на вводе, но \t{X}, а также все точки по пути заменяются плюсами \t{+}.

\Example

\begin{example}
\exmp{
5
...X.
.....
O.OOO
.....
....@
}{
Y
..++.
.++..
O+OOO
.++++
....@
}%
\exmp{
5
..X..
.....
OOOOO
.....
..@..
}{
N
}%
\end{example}

\end{problem}
