\begin{problem}{Переместить в начало}{movetofront.in}{movetofront.out}{2 секунды}

% Author: Andrew Stankevich
% Перевод: Сергей Мельников

Вам дан массив $a_1 = 1, a_2 = 2, \ldots, a_n = n$ и последовальность операций: 
переместить элементы с $l_i$ по $r_i$ в начало массива.
%You have an array $a_1 = 1, a_2 = 2, \ldots, a_n = n$
%and a sequence of queries: move elements from $l_i$ to $r_i$ 
%to front.
Например, для массива $2, 3, 6, 1, 5, 4$, после операции
$(2, 4)$ новый порядок будет $3, 6, 1, 2, 5, 4$.
%For example, if the array is $2, 3, 6, 1, 5, 4$, after the query 
%$(2, 4)$ the new order of elements in the array is $3, 6, 1, 2, 5, 4$. 
А после применения операции $(3, 4)$ порядок элементов в массиве будет $1, 2, 3, 6, 5, 4$.
%If, for example, the query $(3, 4)$ follows,
%the new order of elements is $1, 2, 3, 6, 5, 4$.

Выведите порядок элементов в массиве после выполнения всех операций.
%Print the final order of elements in the array.

\InputFile
В первой строке входного файла указаны числа $n$ и $m$
($2 \le n \le 100\,000$, $1 \le m \le 100\,000$) --- число элементов в массиве и число операций.
Следующие $m$ строк содержат операции в виде двух целых чисел:
$l_i$ и $r_i$ ($1 \le l_i \le r_i \le n$).

%The first line of the input file contains two integer numbers
%$n$ and $m$ ($2 \le n \le 100\,000$, $1 \le m \le 100\,000$)~---
%the number of elements and the number of queries. The
%following $m$ lines contain queries, each line contains two
%integer numbers $l_i$ and $r_i$ ($1 \le l_i \le r_i \le n$).

\OutputFile
Выведите $n$ целых чисел~--- порядок элементов в массиве после применения всех операций.
%Output $n$ integer numbers --- the order of elements in the 
%final array, after executing all queries.

\Example

\begin{example}
\exmp{
6 3
2 4
3 5
2 2
}{
1 4 5 2 3 6
}%
\end{example}

\end{problem}

