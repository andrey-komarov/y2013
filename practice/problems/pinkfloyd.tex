\begin{problem}{Pink Floyd}{floyd.in}{floyd.out}{1 секунда}{}

Группа \emph{Pink Floyd} собирается дать новый концертный тур по 
всему миру. По предыдущему опыту группа знает, что солист 
\emph{Роджер Уотерс} постоянно нервничает при перелетах. На некоторых 
маршрутах он теряет вес от волнения, а на других~--- много ест и 
набирает вес. 

Известно, что чем больше весит Роджер, тем лучше выступает группа, 
поэтому требуется спланировать перелеты так, чтобы вес Роджера на 
каждом концерте был максимально возможным.

Группа должна посещать города в том же порядке, в котором она дает 
концерты. При этом между концертами группа может посещать
промежуточные города.

\InputFile

Первая строка входного файла содержит три натуральных числа $n$, 
$m$ и $k$~--- количество городов в мире, количество рейсов и 
количество концертов, которые должна дать группа соответственно 
($n \leqslant 100$, $m \leqslant 10\,000$, $2 \leqslant k \leqslant 10\,000$). Города 
пронумерованы числами от $1$ до $n$.

Следующие $m$ строк содержат описание рейсов, по одному на строке. 
Рейс номер~$i$ описывается тремя числами $b_i$, $e_i$ и $w_i$~---
номер начального и конечного города рейса и предполагаемое изменение 
веса Роджера в миллиграммах ($1 \leqslant b_i, e_i \leqslant n$, 
$-100\,000 \leqslant w_i \leqslant 100\,000$).

Последняя строка содержит числа $a_1, a_2, ..., a_k$~--- номера городов, в 
которых проводятся концерты ($a_i \neq a_{i+1}$). В начале концертного тура 
группа находится в городе $a_1$.

Гарантируется, что группа может дать все концерты.

\OutputFile        

Первая строка выходного файла должна содержать число $l$~--- 
количество рейсов, которые должна сделать группа. Вторая строка 
должна содержать $l$ чисел~--- номера используемых рейсов.

Если существует такая последовательность маршрутов между концертами, 
что Роджер будет набирать вес неограниченно, то первая строка выходного 
файла должна содержать строку ``\texttt{infinitely kind}''.

\Example

\begin{example}
\exmp{
4 8 5
1 2 -2
2 3 3
3 4 -5
4 1 3
1 3 2
3 1 -2
3 2 -3
2 4 -10
1 3 1 2 4
}{
6
5 6 5 7 2 3
}%
\exmp{
4 8 5
1 2 -2
2 3 3
3 4 -5
4 1 3
1 3 2
3 1 -2
3 2 -3
2 4 10
1 3 1 2 4
}{
infinitely kind
}%
\end{example}

\end{problem}
