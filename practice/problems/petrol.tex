\begin{problem}{Заправки}{petrol.in}{petrol.out}{1 секунда}{64 мегабайта}

В стране $N$ городов, некоторые из которых соединены между собой догогами.
Для того, чтобы проехать по одной дороге, требуется один бак бензина.
В каждом городе бак бензина имеет разную стоимость. Вам требуется добраться из первого города в $N$-ый, потратив как можно меньшее количество денег.

\InputFile

Во входном файле записано сначала число $N$ ($1 \le N \ne 100$), 
затем идёт $N$ чисел, $i$-е из которых
задаёт стоимость бензина в $i$-м городе
(всё это целые числа из диапазона от 0 до 100).
Затем идёт число $M$~--- количество дорог в стране, далее идёт описание самих
дорог.
Каджа дорога задаётся двумя числами~--- номерами городов, которые она соединяет.
Все дороги двусторонние (то есть, по ним можно ездить как в одну,
так и в другую сторону), между двумя городами всегда существует не более одной
дороги, не существует дорог, ведущих из города в себя.

\OutputFile

В выходной файл выведите одно число~--- суммарную стоимость маршрута или $-1$,
если добраться невозможно.

\Example

\begin{example}
\exmp{
4
1 10 2 15
4
1 2 1 3 4 2 4 3
}{
3
}%
\exmp{
4
1 10 2 15
0
}{
-1
}%
\end{example}

\end{problem}

